% !TeX document-id = {5c05c331-134f-48f1-8db3-fd32c67a647b}
%%%%%%%%%%%%%%%%%%%%%%%%%%%%%%%%%%%%%%%%%
% Journal Article
% LaTeX Template
% Version 2.0 (February 7, 2023)
%
% This template originates from:
% https://www.LaTeXTemplates.com
%
% Author:
% Vel (vel@latextemplates.com)
%
% License:
% CC BY-NC-SA 4.0 (https://creativecommons.org/licenses/by-nc-sa/4.0/)
%
% NOTE: The bibliography needs to be compiled using the biber engine.
%
%%%%%%%%%%%%%%%%%%%%%%%%%%%%%%%%%%%%%%%%%

% Magic comments for TeXStudio
% !TeX program = pdflatex
% !BIB program = biber
% !TeX encoding = utf8
% !TeX spellcheck = en_US

%----------------------------------------------------------------------------------------
%	PACKAGES AND OTHER DOCUMENT CONFIGURATIONS
%----------------------------------------------------------------------------------------

\documentclass[
	a4paper, % Paper size, use either a4paper or letterpaper
	10pt, % Default font size, can also use 11pt or 12pt, although this is not recommended
	unnumberedsections, % Comment to enable section numbering
	twoside, % Two side traditional mode where headers and footers change between odd and even pages, comment this option to make them fixed
]{LTJournalArticle}

\bibliography{bibAlessia.bib} % BibLaTeX bibliography file
\bibliography{bibAntoine.bib} % BibLaTeX bibliography file
\bibliography{bibNathan.bib} % BibLaTeX bibliography file

\runninghead{From Finger Stick Blood Glucose Samples to the Predictive Power of Continuous Glucose Monitoring} % A shortened article title to appear in the running head, leave this command empty for no running head

\footertext{} % Text to appear in the footer, leave this command empty for no footer text

\setcounter{page}{1} % The page number of the first page, set this to a higher number if the article is to be part of an issue or larger work

% Main header file
\include{header/additional-packages}

%----------------------------------------------------------------------------------------
%	TITLE SECTION
%----------------------------------------------------------------------------------------

\title{From Finger Stick Blood Glucose Samples to the Predictive Power of Continuous Glucose Monitoring} % Article title, use manual lines breaks (\\) to beautify the layout

% Authors are listed in a comma-separated list with superscript numbers indicating affiliations
% \thanks{} is used for any text that should be placed in a footnote on the first page, such as the corresponding author's email, journal acceptance dates, a copyright/license notice, keywords, etc
\author{%
	Nathan Hoffman\textsuperscript{1}\thanks{Corresponding author: \href{mailto:nathan.hoffman@students.unibe.ch}{nathan.hoffman@students.unibe.ch} \\ \textbf{Received:} September 23, 2024, \textbf{Published:} \today}, Antoine Biebuyck\textsuperscript{1}, Alessia Bruzzo\textsuperscript{1} \\ 
}

% Affiliations are output in the \date{} command
\date{\footnotesize\textsuperscript{\textbf{1}}ARTORG Center for Biomedical Engineering Research, University of Bern, Bern, Switzerland}

% Full-width abstracta
\renewcommand{\maketitlehookd}{%
	\begin{abstract}
		\noindent With the current ageing population, the prevalence of Parkinson's disease, an age-related disorder, is significantly increasing. More research is required to deepen our understanding of this disease with the goal of developing new diagnostic methods, monitoring disease progression and uncovering new potential treatments. Gait analysis has been previously suggested as an effective tool for studying Parkinson's disease. There are many modalities that can be used to examine gait; however, accelerometers present themselves as a leading candidate in clinical applications, given their lower cost and ability for continuous tracking outside of a laboratory setting. In a first part, this study validates the use of accelerometers to count steps by comparing measured data with an actual known amount of steps. Subsequently, several temporal gait parameters from healthy participants and a Parkinson's disease patient were calculated using accelerometer data and shown to differ.
	\end{abstract}
}

%----------------------------------------------------------------------------------------

\begin{document}

\maketitle % Output the title section

%----------------------------------------------------------------------------------------
%	ARTICLE CONTENTS
%----------------------------------------------------------------------------------------

% !TeX encoding = utf8
% !TeX spellcheck = en_US

\section{Introduction}
\begin{comment}
%$\text{C}_6\text{H}_{12}\text{O}_6$, a molecule known as glucose, is the product of the digestive breakdown of carbohydrates, which alongside lipids and proteins, fuel the metabolic processes of the human body \cite{noauthor_human_nodate}. Glucose is distributed throughout the body and delivered to the various tissues and cells via the bloodstream while the pancreas, by using the hormones insulin and glucagon, carefully works to maintain a healthy glucose concentration in the blood \cite{roder_pancreatic_2016}.
%Diabetes, categorized as either type I or type II, is a chronic endocrine disease in which the body can no longer control the proper homeostasis of blood glucose concentration, leading to hypo- and hyperglycemia. Type I diabetes occurs when the pancreas produces insufficient or even no insulin at all. In contrast, type II diabetes is marked by the body's resistance to and inability to effectively use insulin \cite{world_health_organization_global_2016}. Recent estimates reveal that roughly 529 million people worldwide are suffering from diabetes, highlighting the magnitude of this disease \cite{ong_global_2023}.

Health complications, morbidity and mortality are significant risks plaguing individuals living with diabetes. These include, for example, loss of vision, nerve damage, end-stage renal disease, higher rates of cardiovascular events such as stroke and myocardial infarction, increased rate of cancer, increased rates of physical and cognitive disability and premature death \cite{world_health_organization_global_2016}. Hence, it is imperative for diabetics to keep their blood glucose levels within the healthy range, considered to be 70 to 180 mg/dL \cite{noauthor_time_2021}.
%Old sentence
%Fortunately, individuals are able to live healthy lives and reduce their number of complications through proper and robust management.
Proper and robust management, allowing for this glycemic control, ensures that patients can minimize complications and lead healthy lives. Being distinct facets of diabetes, type I and type II have slightly heterogeneous management strategies. For type I, patients require daily administration of exogenous insulin, with the dosage adjusted according to carbohydrate intake and exercise, to mimic the insulin secretion pattern in the absence of disease. For type II, the treatment regiment is more case-dependent, as initial consultation may focus on lifestyle changes for weight-reduction while second line therapies include drugs and oral medications to reduce hepatic glucose production or to reduce insulin resistance \cite{bilous_handbook_2021}. Insulin therapy, as described for type I patients, may also be a treatment option for type II patients \cite{noauthor_patient_nodate}. 
%Old sentence
%In the specific case of blood glucose, on top of monitoring the levels, they need to interpret such complex patterns . 
% I DON'T THINK ALL THE TALK ABOUT ARTIFICIAL INTELLIGENCE IS NECESSARY.
To this end, diabetes is a disease that requires the patient to take an active role, a process referred to as self-management. This brings about additional burden and emotional distress which can result in sub-optimal management \cite{adu_enablers_2019}. In the particular instance of blood glucose, not only do patients need to monitor their levels but they also need to continuously understand and interpret their patterns, a task that is already challenging enough for experienced clinicians given the shear amount of data and its seeming stochasticity. Therefore, it is clear that new assistive and sophisticated tools are required, and with the current advances in the field, artificial intelligence presents itself as a leading candidate \cite{mayya_need_2024}.
Artificial intelligence can be used to support many aspects of diabetes care such as assessing risk of developing the disease, diagnosis, lifestyle recommendations among others \cite{mayya_need_2024}. Of interest in this study is the use of artificial intelligence, and so called data-driven methods, in the prediction of blood glucose levels. This subsequently raises the question, given if it is possible to predict future glucose values, what is the prediction horizon, or how far into the future can this prediction still be accurate. Addressing this question is impactful as it could allow for patients using only single finger sticks to have an idea of what their continuous glucose would be or could notify patients using continuous glucose monitoring that they need to take corrective action to avoid an adverse glycemic event.


Continuous glucose monitoring (CGM) systems have revolutionised diabetes care by providing minute-by-minute glucose readings, offering valuable insights into glucose dynamics that enable improved glycemic control and therapy adjustments. Unlike traditional self-monitoring of blood glucose (SMBG) methods, which rely on intermittent measurements such as finger prink samples, CGMs offer real-time trend analysis, allowing patients and clinicians to proactively manage glucose levels and reduce the risk of adverse events \cite{Heinemann2018, Beck2017, Battelino2019}.

\textcolor{red}{TO ALL : let me know if you agree with this: }
Despite these advantages, CGMs are not universally accessible, particularly for individuals with T2D, where SMBG remains the predominant method of glucose monitoring\textcolor{red}{ add reference here }. This project seeks to bridge this gap by investigating the minimum number of SMBG measurements required to predict future glucose values with accuracy comparable to CGMs. 

By leveraging \textcolor{red}{put here the method we ended up using }, this project evaluates whether subsampling CGM data—simulating SMBG measurements—can still provide reliable predictions of future blood glucose levels. 


\end{comment}

Diabetes is a chronic metabolic disease caused by insufficient or absent insulin production by the pancreas, leading to elevated blood glucose (BG) levels. Maintaining BG levels within a healthy range (70–180 mg/dL) is critical for individuals with diabetes to prevent acute and chronic complications such as cardiovascular disease, neuropathy, and kidney failure \cite{ADATIR, Roglic2016}. Management strategies vary by diabetes type: individuals with Type 1 Diabetes (T1D) rely on daily insulin administration, while those with Type 2 Diabetes (T2D) often begin with lifestyle interventions before progressing to medications or insulin therapy \cite{Roglic2016}. Regardless of the type, effective management hinges on accurate BG monitoring and prediction to enable timely interventions.

Continuous glucose monitoring (CGM) systems have transformed diabetes care by providing real-time glucose readings, offering insights into glucose dynamics that support improved glycemic control and proactive management of hypo- and hyperglycemic events \cite{Heinemann2018, Beck2017, Battelino2019}. However, CGM adoption remains limited due to economic and systemic barriers, particularly among low-income populations, leaving many individuals reliant on self-monitoring of blood glucose (SMBG) methods such as finger prick samples \cite{Oser2021, ADA}. This disparity highlights the need for innovative approaches to make the benefits of CGM-like data accessible to a broader population.

This project addresses this challenge by investigating the potential of sparse blood glucose measurements, such as SMBG, to approximate the predictive power of CGM systems. Using a Subsample-Reconstruct-Analyze (SRA) framework, the CGM data is systematically subsampled to simulate sparse SMBG-like patterns and applied interpolation techniques (linear and cubic spline) to reconstruct glucose profiles. The fidelity of the reconstructed data is evaluated using mean squared error (MSE) and time-in-range metrics, assessing glucose levels across clinically significant thresholds. This project provides insights into the trade-offs between sampling resolution and data fidelity, offering practical implications for diabetes management and sensor optimisation in resource-constrained settings. 

% !TeX encoding = utf8
% !TeX spellcheck = en_US

\section{Related Work}

Subsampling and reconstruction methods are commonly used in time-series analysis to deal with irregular or incomplete data. In diabetes management, most research focuses on predicting glucose levels using CGM data. However, less attention has been given to sparse measurements like those from SMBG to recreate data similar to CGM.

One method, Kalman smoothing, has been used to process glucose data from irregular sources like SMBG or Flash Glucose Monitors (FGM). This method fills in missing data by creating smooth estimates and providing information about how accurate those estimates are. It also corrects errors in the data, such as outliers, making it particularly useful for looking at past glucose trends \cite{Staal2019}.

Another method, a three-level B-spline model, has been used to analyze CGM data and measure glucose variability. This approach helps compare glucose trends between individuals and across different days. It has shown significant differences in glucose variability between groups and is useful for filling gaps in sparse or irregular data \cite{Zheng2011}.

Building on these methods, this project uses a SRA framework to study how reducing the number of glucose readings affects data accuracy. By applying cubic spline and polynomial interpolation, we explore the balance between fewer data points and the quality of reconstructed glucose profiles.


% !TeX encoding = utf8
% !TeX spellcheck = en_US

\section{Methods}

\subsection{Participants}
Two male participants 25 years of age were recruited by the ARTORG Center for Biomedical Engineering Research. These two participants were healthy and did not suffer from any known pre-existing medical conditions. This study was conducted at the NeuroTec research facility in the Swiss Institute for Translational and Entrepreneurial Medicine (sitem-insel) and at the Stadion Neufeld, both located in Bern, Switzerland.

\subsection{Data collection}
The AX6 Axivity 6-axis logging movement sensor (Newcastle, United Kingdom) was used to collect accelerometer data in this study. The puck has dimensions of 23 $\times$ 32.5 $\times$  8.9\,mm and a weight of 11\,g.

On both the left and right legs, the sensors were placed on the anterior aspect of the tibia, 7\,cm above the lateral malleolus as depicted in Figure \ref{fig:sensorplacement}. To affix the sensors, a one-sided adhesive interface was utilized and the sensor was placed beneath this adhesive. 

Positioning was selected so that the walking direction would follow the z-direction, as depicted in Figure \ref{fig:sensorplacement}, and based on findings from a prior systematic review indicating that IMU placement closer to the ground enhances gait event detection \cite{pacini_panebianco_analysis_2018}.

\begin{figure}[h]
	\centering
	\includegraphics[width=0.6\linewidth]{"Figures/SensorPlacement"}
	\caption{Sensor Placement \cite{noauthor_foot_nodate}}
	\label{fig:sensorplacement}
\end{figure}

After sensor placement, the participants were asked to walk, at what they consider their regular walking speed, one lap around a 400\,m track. Before starting and at the end of the walk, the participants were asked to jump in place three times. The idea was to set reference points to facilitate later analysis since this would create evident signals in the accelerometer data.

The step counter IOS app Counter+ developed by Yan Kin LEUNG was used to establish a ground truth of the actual number of steps walked during the experiment. The participants were asked to use this application to count their steps while walking. Additionally, the participants were followed by a researcher also counting their steps using the same application. Importantly, the follower carefully remained behind the participant and out of sight in order to reduce any chance of influencing the participant's gait. The mean from both counts was taken and is displayed in Table \ref{tab:results} in the row named \emph{counted total steps}.

\subsection{Parkinson's disease patient data}
In collaboration with the Gerontechnology and Rehabilitation Research Group of the ARTORG Center for Biomedical Engineering Research and the Division of Cognitive and Restorative Neurology within the medical faculty at the University of Bern, we were kindly provided with previously collected accelerometer data from a patient suffering from Parkinson's disease. 

The Parkinson's disease patient's data was recorded over a five minute time period during which the patient walked back and forth in a 30\,m long hallway. Due to these changes in direction, the signal was very inconsistent and discontinuously periodic. Therefore, to facilitate the analysis, only a 24 second interval from the signal was selected.


\subsection{Data processing}
The accelerometers were setup and configured using the AX3/AX6 OMGUI Configuration and Analysis Tool which is an open source application developed by the Open Movement Team at Newcastle University, United Kingdom. This application allowed the data to be exported as CSV files which was later processed with Python v3.12. Additional packages, as displayed in Table \ref{tab:packages}, were used to further process the data.

\begin{table}[H] % Full width table (notice the starred environment)
	\caption{Additional packages for data processing.}
	\centering % Horizontally center the table
	\begin{tabularx}{\linewidth}{X|X|} % Manually specify column alignments with L{}, R{} or C{} and widths as a fixed amount, usually as a proportion of \linewidth
		\toprule
		Package & Version \\
		\midrule
		Matplotlib & 3.8.0 \\
		NumPy & 1.26\\
		Pandas & 2.2.1\\
		SciPy & 1.12.0\\
		Sensor Motion & 1.1.4\\
		\bottomrule
	\end{tabularx}
	\label{tab:packages}
\end{table} 

\subsection{Filtering}
In order to reduce the noise, the accelerometer data was filtered with a third-order low-pass Butterworth filter with a 5\,Hz cutoff frequency. In order to remove the DC component of the signal, a third-order high-pass Butterworth filter with a 0.5\,Hz cutoff frequency was applied. Together, these filters band-passed the signal. These cutoff values were chosen so that the three greatest amplitude frequencies were contained in the smoothed signal. This was done by taking the Fast Fourier Transform (FFT) of the signal to go from the time domain to the frequency domain. 

For visualization purposes, sample FFT plots, before and after filtering, can be seen in Figure \ref{fig:fft_plot}. The sampling rate of the accelerometers was kept at the default value of 100\,Hz, which is adequate for studies of human movement. Note the 50\,Hz frequency axis due to the Nyquist-Shannon sampling theorem which states that the sampling rate must be two times the largest frequency \cite{mcclellan_dsp_2017}.

\begin{figure}[H]
	\centering
	\begin{subfigure}{\linewidth}
		\centering
		\includegraphics[width=\linewidth]{Figures/fft_plot.png}
		\caption{FFT plot before filtering.}
		\label{fig:fft_plot_before}
	\end{subfigure}\vfill
	\begin{subfigure}{\linewidth}
		\centering
		\includegraphics[width=\linewidth]{Figures/fft_after_filter_plot.png}
		\caption{FFT plot after filtering.}
		\label{fig:fft_plot_after}
	\end{subfigure}
	\caption{FFT plots of the signal from participant B.}
	\label{fig:fft_plot}
\end{figure}

\subsection{Peak detection}
The definitions for initial contact (IC) and terminal contact (TC) provided by Gottlied and colleagues were used. Initial contact is defined as the moment in the gait cycle when the foot first makes contact with the ground, corresponding to heel-strike during forward walking (FW). Terminal contact is defined as the instant in the gait cycle when the foot completely lifts off the ground, corresponding to toe-off during FW \cite{gottlieb_agreement_2020}.
\begin{figure}[h]
	\centering
	\includegraphics[width=\linewidth]{Figures/peak_detection.png}
	\caption{Peak and valley detection for participant B.}
	\label{fig:peak_detection}
\end{figure}
The \emph{find\_peaks} method, available in the Sensor Motion Python package, was used to detect peaks in the accelerometer signal data. This method employs advanced algorithms to identify significant changes or peaks within the signal, crucial for recognizing distinct motion patterns or events. By specifying parameters such as minimum peak height or distance between peaks (number of samples), the peak detection process was tailored to suit the specific analysis requirements. The values selected for these parameters can be found in Table \ref{tab:peak_values}.
\begin{table}[H] % Full width table (notice the starred environment)
	\caption{Values for peak and valley detection.}
	\centering % Horizontally center the table
	\begin{tabularx}{\linewidth}{l|c|c|c|} % Manually specify column alignments with L{}, R{} or C{} and widths as a fixed amount, usually as a proportion of \linewidth
		\toprule
		& Parkinson's  & Part. A & Part. B \\
		\cmidrule(r){2-4}
		Distance & 75 & 50 & 50\\
		Peak height& 0.50 & 0.75 & 0.65\\
		Valley height & 0.50 & 0.68 & 0.70\\
		\bottomrule
	\end{tabularx}
	\label{tab:peak_values}
\end{table} 

Additionally, visual inspection, as shown in Figure \ref{fig:peak_detection}, was necessary to confirm the success of the method. This approach facilitated the extraction of key features from the accelerometer data, enabling the characterization and interpretation of various motion-related phenomena with precision and accuracy. 

\subsection{Calculation of temporal gait parameters}
Once the peaks were obtained, calculating steps is quite straightforward since it entails just counting the number of peaks found.

Zanardi and colleagues describe certain temporal gait parameters in which a meta-analysis showed a difference between Parkinson's disease patients and healthy controls \cite{zanardi_gait_2021}. Therefore, given what was attainable from the obtained data, some of these parameters, such as mean swing time, stance time, stride time and cadence were calculated. The calculations for these parameters can be found in Table \ref{tab:calculaitons} \cite{grucci_gait_2019}.

 \begin{table}[H] % Full width table (notice the starred environment)
 	\caption{Calculations for the temporal parameters which include IC and TC of the foot that is leading (IC1 and TC1) and the foot that is lagging (IC2 and TC2).}
 	\centering % Horizontally center the table
 	\begin{tabularx}{\linewidth}{X|c|c|} % Manually specify column alignments with L{}, R{} or C{} and widths as a fixed amount, usually as a proportion of \linewidth
 		\toprule
 		Temporal parameters & Leading & Lagging \\
 		\midrule
 		Swing time (s) & IC1 - TC1 & IC2 - TC2\\
 		Stance time (s)  & TC1 - IC1 & TC2 - IC2\\
 		Stride time (s)  & IC1 - IC1 & IC2 - IC2\\
 		Cadence (steps/min) & \multicolumn{2}{c|}{2*60/(mean stride time)}\\
 		\bottomrule
 	\end{tabularx}
 	\label{tab:calculaitons}
 \end{table} 

Swing time is defined as the duration of when the reference limb is not in contact with the ground whereas, conversely, during stance time, the reference limb is in contact with the ground. Stride time is defined as the duration between the initial contact of a foot and the ensuing initial contact of the ipsilateral foot. Cadence, as the units may suggest, is the number of steps taken during an interval of time, which is per minute in this case \cite{webster_principles_2019}.


%Not going into report but leaving for information
%Some important gait parameters are speed, stride length, cadence, step width, single and double support time, swing time, range of motion for the hip, knee and ankle and the angle at initial contact for the hip, knee and ankle \cite{zanardi_gait_2021}.

\subsection{Removal of Outliers}
In full transparency, during the analysis of the gait data, outliers were observed and subsequently removed - for swing time, 3 outliers for participant A and 2 outliers for participant B, and for stance time, 2 outliers for both participants.

Outliers can heavily influence summary statistics such as mean, standard deviation, and correlation coefficients. By removing outliers, more accurate estimates of these statistics can be obtained, which in turn lead to more reliable interpretations of the data. As such the samples which were three standard deviations larger or smaller than the mean were removed \cite{osborne_power_nodate}.

\subsection{Accuracy of Step Counter}
In order to get some sort of a metric to describe the validity of accelerometers as a step counter, the error rate was calculated using the Equation \eqref{eqn:accuracyEquation} described by Cleland and colleagues \cite{cleland_effects_2012}.
\begin{equation}
	\label{eqn:accuracyEquation}
	Accuracy = \biggl(1 - \frac{C_s - A_s}{A_s}\biggr) \times 100
\end{equation}
where $C_s$ is the calculated steps from the accelerometer data and $A_s$ is the steps counted, which is taken here as the ground truth. 

















% !TeX encoding = utf8
% !TeX spellcheck = en_US

\section{Data and Experimental Setup}

The data used in this study was procured from the OhioT1DM Dataset. 
















% !TeX encoding = utf8
% !TeX spellcheck = en_US

\section{Results}



% !TeX encoding = utf8
% !TeX spellcheck = en_US

\section{Discussion}
The findings of this study provide critical insights into the trade-offs associated with subsampling rates in continuous glucose monitoring (CGM) data, with implications for both clinical applications and research involving glycemic metrics. This discussion elaborates on the impact of subsampling on key metrics, the fidelity of glucose signal reconstruction, and the clinical relevance of time-in-range metrics.

\subsection{Impact of Subsampling on Mean Blood Glucose and Variability}
The results demonstrate that mean blood glucose values are largely preserved at lower subsampling rates (e.g., rates 1 and 2). However, statistically significant deviations ($p \leq 0.05$) become apparent at higher subsampling rates (e.g., rates 16 and 32), particularly for patients exhibiting higher glycemic variability, such as Patients 563 and 559. These deviations highlight interindividual variability in the robustness of mean glucose metrics under subsampling. This variability likely reflects differences in the temporal patterns of glucose fluctuations, emphasizing the need for individualized considerations when applying data reduction techniques.
Overall, these results suggest that while subsampling minimally affects mean blood glucose values at lower rates, higher subsampling rates may introduce significant deviations, particularly for specific patients.

Furthermore, the observed reduction in standard deviation at higher subsampling rates is particularly noteworthy. While standard deviation—a critical metric for evaluating glucose variability(CITATION NEEDED) —remains stable at lower rates (e.g., rates 1, 2, and 4), substantial reductions are observed at rates of 16 and 32. This trend indicates that extreme subsampling may suppress variability, potentially obscuring clinically meaningful fluctuations in glucose levels. Given the importance of variability metrics (CITATION NEEDED) for assessing glycemic control and predicting adverse outcomes, the findings underscore the limitations of aggressive subsampling in preserving the integrity of these metrics.
% These findings emphasize the trade-offs between data reduction and the preservation of variability in glucose measurements, with implications for accurately assessing glycemic control and variability metrics.

\subsection{Reconstruction Error and Signal Fidelity}
The mean squared error (MSE) analysis, as depicted in Figure~\ref{fig:mses}, reveals that reconstruction fidelity is highly dependent on the subsampling rate. At lower rates (e.g., 1, 2, and 4), MSE values remain low, indicating minimal information loss. However, as the subsampling rate increases to 16 and 32, the MSE rises sharply, reflecting substantial degradation in reconstruction accuracy. This loss of fidelity is further evidenced by the distortion of reconstructed glucose signals at higher rates, as shown in Figure~\ref{fig:SR_combined}. Key features of glucose dynamics, such as peak amplitudes and nadirs, are progressively smoothed or distorted, undermining the ability to accurately capture the temporal complexity of glucose fluctuations.
These results emphasize the trade-off between sampling frequency and reconstruction fidelity, with higher subsampling rates introducing substantial error and variability in capturing blood glucose dynamics.

The comparative performance of cubic spline and degree-5 polynomial interpolation methods highlights the superiority of the former for reconstructing glucose signals, particularly at lower subsampling rates. Although the performance gap narrows at higher rates, the consistently lower MSE values for cubic splines support its use as the preferred method for CGM data reconstruction. The widening variability in MSEs at higher subsampling rates also underscores the heterogeneity in reconstruction accuracy across patients, suggesting that some individuals may be more susceptible to information loss under data reduction.

\subsection{Glycemic Ranges and Clinical Implications}
The analysis of time spent in different glycemic ranges further elucidates the impact of subsampling on clinically relevant metrics. For most patients, the time spent in hypoglycemia (\textless 70~mg/dL) and the target range (70--180~mg/dL) remains stable across lower subsampling rates (e.g., 1, 2, and 4). However, deviations emerge at higher rates, particularly for patients with greater glycemic variability or those spending significant time in extreme glucose ranges.

The reduction in the time spent in hyperglycemia (\textgreater 180~mg/dL) at higher subsampling rates is particularly concerning, as it may lead to an underestimation of prolonged periods of elevated glucose levels. This underestimation is most pronounced in patients with a higher initial percentage of time in hyperglycemia, as observed for specific patients (e.g., orange and blue dots in Figure~\ref{fig:time_in_ranges}). Such distortions could have significant implications for clinical decision-making, as they may compromise the accuracy of assessments regarding glycemic control and treatment efficacy (CITATION NEEDED).

\subsection{Balancing Data Reduction and Analytical Integrity}
These findings emphasize the trade-offs between data reduction and the preservation of clinically meaningful metrics in CGM data. While subsampling at lower rates (e.g., 1, 2, and 4) appears to maintain the integrity of mean values, variability metrics, and time-in-range distributions, higher rates introduce substantial reconstruction errors and distort key glycemic metrics. Thus, the selection of subsampling rates should be guided by the specific analytical objectives and the need to balance computational efficiency with the preservation of data fidelity.

\subsection{Real-World Constraints and Practical Considerations}
The findings of this study highlight the potential trade-offs between data fidelity and practical constraints in continuous glucose monitoring (CGM). In real-world scenarios, achieving the balance between data accuracy and the limitations of resource availability is critical. High-resolution data collection requires significant storage capacity and computational resources, both of which may pose challenges in resource-constrained environments or in long-term monitoring applications. For instance, patients using CGM devices for months or years may experience difficulties with data storage or retrieval due to the sheer volume of high-frequency recordings.

Moreover, the transmission of high-resolution CGM data, particularly in real-time, may face challenges such as limited bandwidth in telemedicine applications or power constraints in portable devices (CITATION NEEDED). Subsampling techniques could provide a practical solution by reducing the data volume while retaining key trends and patterns, thereby enabling more efficient data transmission and storage. However, the results of this study emphasize that care must be taken to select appropriate subsampling rates to ensure that critical information, particularly related to glycemic variability, is not lost.

These constraints are particularly relevant for remote or resource-limited settings, where access to high-quality monitoring devices and infrastructure may be limited (CITATION NEEDED). Subsampling approaches, when optimized, could make CGM technology more accessible and affordable in such settings (CITATION NEEDED). However, it is important to note that the trade-offs introduced by subsampling must not compromise the accuracy of clinically relevant metrics, such as time in range, or the detection of rapid glucose fluctuations.

Future work could explore adaptive subsampling techniques that dynamically adjust sampling rates based on observed glycemic variability or patient-specific characteristics. Such approaches may offer a more tailored balance between minimizing data volume and preserving the integrity of critical glucose measurements. Additionally, advancements in data compression and signal processing algorithms could further alleviate the constraints associated with high-frequency data collection, offering alternative pathways for overcoming the limitations observed in this study.

\subsection{Limitations and Future Work}
\subsubsection{Limitations}
While this study provides valuable insights into the impact of subsampling on blood glucose monitoring, several limitations warrant consideration. First, the analysis is based on a specific dataset and may not fully capture the diversity of glycemic patterns observed across different populations, such as pediatric patients or individuals with distinct metabolic disorders. The observed variability in subsampling effects across patients underscores the importance of validating these findings across larger and more diverse cohorts.

Second, this study assumes consistent device performance and signal quality at all sampling rates. In real-world CGM systems, noise, calibration inaccuracies, and sensor drift can influence the accuracy of both raw and subsampled data. These factors, not addressed explicitly in this analysis, may amplify the errors observed at higher subsampling rates. Future studies should incorporate realistic noise models to evaluate the robustness of subsampling techniques under variable signal quality.

Additionally, the use of cubic spline interpolation for data reconstruction, while effective, represents only one of many possible methods. The performance of alternative interpolation techniques, such as machine learning-based approaches or adaptive filtering methods, remains unexplored and could yield different outcomes. Furthermore, the current study evaluates subsampling based on fixed rates, which may not fully reflect real-world scenarios where glycemic variability and clinical context could influence sampling needs.

\subsubsection{Future Work}
Building on the findings of this study, future research could explore several promising directions:

\begin{itemize}
	\item \textbf{Adaptive Subsampling Strategies:} One key area of interest is the development of adaptive subsampling techniques that dynamically adjust sampling rates based on glycemic variability or patient-specific factors. Such strategies could optimize data acquisition by focusing on critical periods, such as times of rapid glucose fluctuations, while reducing sampling during stable periods.
	
	% \item \textbf{Noise and Signal Quality Modeling:} Incorporating realistic models of sensor noise and device errors would provide a more comprehensive evaluation of subsampling techniques under real-world conditions. This could help determine the thresholds at which subsampling becomes clinically unacceptable.
	
	\item \textbf{Alternative Reconstruction Methods:} Exploring advanced interpolation and reconstruction techniques, such as machine learning algorithms or hybrid methods, may improve the accuracy of data reconstruction, particularly at higher subsampling rates. Comparing the performance of these methods against traditional techniques like cubic splines could yield insights into their practical applicability.
	
	\item \textbf{Clinical Outcomes:} While this study focuses on the accuracy of reconstructed glucose data, future work could assess the clinical implications of subsampling on decision-making and patient outcomes. For example, the impact of subsampling on insulin dosing recommendations, hypoglycemia prediction, and time-in-range metrics should be systematically evaluated.
	
	\item \textbf{Real-Time Applications:} In scenarios where data transmission bandwidth or device power is limited, subsampling could be implemented in real-time. Investigating the feasibility and effectiveness of real-time adaptive subsampling in CGM devices could bridge the gap between theoretical findings and practical applications.
	
	\item \textbf{Longitudinal Studies:} Long-term studies are needed to assess the impact of subsampling on extended glucose monitoring periods. These studies should examine the trade-offs between reduced data volume and accuracy over weeks or months of continuous monitoring, providing insights into the feasibility of subsampling in chronic disease management.
	
	\item \textbf{Personalized Subsampling:} Future research could investigate personalized subsampling strategies tailored to individual glycemic patterns and lifestyle factors. Such approaches could enhance the usability and accuracy of CGM systems by accounting for patient-specific variability in glucose dynamics.
\end{itemize}

% !TeX encoding = utf8
% !TeX spellcheck = en_US

\section{Conclusion}

%In summary, while this study demonstrates the potential trade-offs associated with subsampling, addressing the identified limitations and pursuing the outlined future directions could significantly enhance the applicability and reliability of subsampling techniques in blood glucose monitoring. These efforts will be essential to ensure that CGM systems continue to provide high-quality data while meeting the practical constraints of real-world applications.
 
 This study artificially generated sparse glucose measurement data by subsampling, at different rates, continuous glucose monitoring data obtained from patients with T1D. This study investigated whether simple interpolation techniques, such as cubic spline and polynomial of degree five, were capable of reliably reconstructing and approximating the original signal. While achievable at an interval of 40 minutes, this timeframe is still too short to allow fingerstick sample data to compare to that of continuous glucose monitoring.
 
% !TeX encoding = utf8
% !TeX spellcheck = en_US

\section{Acknowledgment}
We would like to thank Maria Panagiotou and Knut Strommen for guiding us throughout this project and for giving us an instructive experience. We wish them the best of luck for their PhDs.


%----------------------------------------------------------------------------------------
%	 REFERENCES
%----------------------------------------------------------------------------------------

\printbibliography % Output the bibliography

%----------------------------------------------------------------------------------------

\end{document}
