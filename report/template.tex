% !TeX document-id = {5c05c331-134f-48f1-8db3-fd32c67a647b}
%%%%%%%%%%%%%%%%%%%%%%%%%%%%%%%%%%%%%%%%%
% Journal Article
% LaTeX Template
% Version 2.0 (February 7, 2023)
%
% This template originates from:
% https://www.LaTeXTemplates.com
%
% Author:
% Vel (vel@latextemplates.com)
%
% License:
% CC BY-NC-SA 4.0 (https://creativecommons.org/licenses/by-nc-sa/4.0/)
%
% NOTE: The bibliography needs to be compiled using the biber engine.
%
%%%%%%%%%%%%%%%%%%%%%%%%%%%%%%%%%%%%%%%%%

% Magic comments for TeXStudio
% !TeX program = pdflatex
% !BIB program = biber
% !TeX encoding = utf8
% !TeX spellcheck = en_US

%----------------------------------------------------------------------------------------
%	PACKAGES AND OTHER DOCUMENT CONFIGURATIONS
%----------------------------------------------------------------------------------------

\documentclass[
	a4paper, % Paper size, use either a4paper or letterpaper
	10pt, % Default font size, can also use 11pt or 12pt, although this is not recommended
	unnumberedsections, % Comment to enable section numbering
	twoside, % Two side traditional mode where headers and footers change between odd and even pages, comment this option to make them fixed
]{LTJournalArticle}

\bibliography{bibAlessia.bib} % BibLaTeX bibliography file
\bibliography{bibAntoine.bib} % BibLaTeX bibliography file
\bibliography{bibNathan.bib} % BibLaTeX bibliography file

\runninghead{From Finger Stick Blood Glucose Samples to the Predictive Power of Continuous Glucose Monitoring} % A shortened article title to appear in the running head, leave this command empty for no running head

\footertext{} % Text to appear in the footer, leave this command empty for no footer text

\setcounter{page}{1} % The page number of the first page, set this to a higher number if the article is to be part of an issue or larger work

% Main header file
% % % % % % % % % % % % % % % % % % % % % % % % % % % % % % % % % % % % % % % % %
%
% OSTReport -- Additional packages frequently used in reports
%
% % % % % % % % % % % % % % % % % % % % % % % % % % % % % % % % % % % % % % % % %

% Mathematical equations
\usepackage{amsmath}
\usepackage{amssymb}
\usepackage{bm}
\usepackage{MnSymbol}
%\usepackage{breqn}

% Tables
\usepackage{multirow}
\usepackage{tabularx}
\usepackage{booktabs}

% Figures
\usepackage{pdfpages}
\usepackage{epstopdf}
\usepackage{float}
\usepackage{graphicx}
\usepackage{caption}
\usepackage{subcaption}
%\usepackage[outdir=./]{epstopdf}

% Quotation marks
\usepackage{csquotes}
\setquotestyle[quotes]{german}

% Si Units
\usepackage{siunitx}
\sisetup{detect-all,sticky-per,per-mode=symbol}

% Multicolumn documents and sections
\usepackage{multicol}

\PassOptionsToPackage{svgnames,x11names,dvipsnames}{xcolor}
\usepackage[most]{tcolorbox}

%----------------------------------------------------------------------------------------
%	TITLE SECTION
%----------------------------------------------------------------------------------------

\title{From Finger Stick Blood Glucose Samples to the Predictive Power of Continuous Glucose Monitoring} % Article title, use manual lines breaks (\\) to beautify the layout

% Authors are listed in a comma-separated list with superscript numbers indicating affiliations
% \thanks{} is used for any text that should be placed in a footnote on the first page, such as the corresponding author's email, journal acceptance dates, a copyright/license notice, keywords, etc
\author{%
	Nathan Hoffman\textsuperscript{1}\thanks{Corresponding author: \href{mailto:nathan.hoffman@students.unibe.ch}{nathan.hoffman@students.unibe.ch} \\ \textbf{Received:} September 23, 2024, \textbf{Published:} \today}, Antoine Biebuyck\textsuperscript{1}, Alessia Bruzzo\textsuperscript{1} \\ 
}

% Affiliations are output in the \date{} command
\date{\footnotesize\textsuperscript{\textbf{1}}ARTORG Center for Biomedical Engineering Research, University of Bern, Bern, Switzerland}

% Full-width abstracta
\renewcommand{\maketitlehookd}{%
	\begin{abstract}
		\noindent While the development of continuous glucose montoring sensors has aided patients with the self-management of their diabetes, it has also created disparaties in health for the patients unable to access the devices due to socioeconomic status. These patients still rely on finger prick samples and so the question arises if it is possible to create continuous glucose montoring-like data from these individual measurements. Therefore, this study subsamples continuous data and applies interpolation techniques to reconstruct the original signal and evaluates whether clinically relevant features can be preserved. The overall finding was that these techniques are able to produce satisfactory reconstructions up to a time interval of 40 minutes. It was deemed that this result was not substantial enough to truly benefit the disfavored patients, consequently, more research is needed to tackle this health inequality. 
	\end{abstract}
}

%----------------------------------------------------------------------------------------

\begin{document}

\maketitle % Output the title section

%----------------------------------------------------------------------------------------
%	ARTICLE CONTENTS
%----------------------------------------------------------------------------------------

% !TeX encoding = utf8
% !TeX spellcheck = en_US

\section{Introduction}
\begin{comment}
%$\text{C}_6\text{H}_{12}\text{O}_6$, a molecule known as glucose, is the product of the digestive breakdown of carbohydrates, which alongside lipids and proteins, fuel the metabolic processes of the human body \cite{noauthor_human_nodate}. Glucose is distributed throughout the body and delivered to the various tissues and cells via the bloodstream while the pancreas, by using the hormones insulin and glucagon, carefully works to maintain a healthy glucose concentration in the blood \cite{roder_pancreatic_2016}.
%Diabetes, categorized as either type I or type II, is a chronic endocrine disease in which the body can no longer control the proper homeostasis of blood glucose concentration, leading to hypo- and hyperglycemia. Type I diabetes occurs when the pancreas produces insufficient or even no insulin at all. In contrast, type II diabetes is marked by the body's resistance to and inability to effectively use insulin \cite{world_health_organization_global_2016}. Recent estimates reveal that roughly 529 million people worldwide are suffering from diabetes, highlighting the magnitude of this disease \cite{ong_global_2023}.

Health complications, morbidity and mortality are significant risks plaguing individuals living with diabetes. These include, for example, loss of vision, nerve damage, end-stage renal disease, higher rates of cardiovascular events such as stroke and myocardial infarction, increased rate of cancer, increased rates of physical and cognitive disability and premature death \cite{world_health_organization_global_2016}. Hence, it is imperative for diabetics to keep their blood glucose levels within the healthy range, considered to be 70 to 180 mg/dL \cite{noauthor_time_2021}.
%Old sentence
%Fortunately, individuals are able to live healthy lives and reduce their number of complications through proper and robust management.
Proper and robust management, allowing for this glycemic control, ensures that patients can minimize complications and lead healthy lives. Being distinct facets of diabetes, type I and type II have slightly heterogeneous management strategies. For type I, patients require daily administration of exogenous insulin, with the dosage adjusted according to carbohydrate intake and exercise, to mimic the insulin secretion pattern in the absence of disease. For type II, the treatment regiment is more case-dependent, as initial consultation may focus on lifestyle changes for weight-reduction while second line therapies include drugs and oral medications to reduce hepatic glucose production or to reduce insulin resistance \cite{bilous_handbook_2021}. Insulin therapy, as described for type I patients, may also be a treatment option for type II patients \cite{noauthor_patient_nodate}. 
%Old sentence
%In the specific case of blood glucose, on top of monitoring the levels, they need to interpret such complex patterns . 
% I DON'T THINK ALL THE TALK ABOUT ARTIFICIAL INTELLIGENCE IS NECESSARY.
To this end, diabetes is a disease that requires the patient to take an active role, a process referred to as self-management. This brings about additional burden and emotional distress which can result in sub-optimal management \cite{adu_enablers_2019}. In the particular instance of blood glucose, not only do patients need to monitor their levels but they also need to continuously understand and interpret their patterns, a task that is already challenging enough for experienced clinicians given the shear amount of data and its seeming stochasticity. Therefore, it is clear that new assistive and sophisticated tools are required, and with the current advances in the field, artificial intelligence presents itself as a leading candidate \cite{mayya_need_2024}.
Artificial intelligence can be used to support many aspects of diabetes care such as assessing risk of developing the disease, diagnosis, lifestyle recommendations among others \cite{mayya_need_2024}. Of interest in this study is the use of artificial intelligence, and so called data-driven methods, in the prediction of blood glucose levels. This subsequently raises the question, given if it is possible to predict future glucose values, what is the prediction horizon, or how far into the future can this prediction still be accurate. Addressing this question is impactful as it could allow for patients using only single finger sticks to have an idea of what their continuous glucose would be or could notify patients using continuous glucose monitoring that they need to take corrective action to avoid an adverse glycemic event.


Continuous glucose monitoring (CGM) systems have revolutionised diabetes care by providing minute-by-minute glucose readings, offering valuable insights into glucose dynamics that enable improved glycemic control and therapy adjustments. Unlike traditional self-monitoring of blood glucose (SMBG) methods, which rely on intermittent measurements such as finger prink samples, CGMs offer real-time trend analysis, allowing patients and clinicians to proactively manage glucose levels and reduce the risk of adverse events \cite{Heinemann2018, Beck2017, Battelino2019}.

\textcolor{red}{TO ALL : let me know if you agree with this: }
Despite these advantages, CGMs are not universally accessible, particularly for individuals with T2D, where SMBG remains the predominant method of glucose monitoring\textcolor{red}{ add reference here }. This project seeks to bridge this gap by investigating the minimum number of SMBG measurements required to predict future glucose values with accuracy comparable to CGMs. 

By leveraging \textcolor{red}{put here the method we ended up using }, this project evaluates whether subsampling CGM data—simulating SMBG measurements—can still provide reliable predictions of future blood glucose levels. 


\end{comment}

Diabetes is a chronic metabolic disease caused by insufficient or absent insulin production by the pancreas, leading to elevated blood glucose (BG) levels. Maintaining BG levels within a healthy range (70–180 mg/dL) is critical for individuals with diabetes to prevent acute and chronic complications such as cardiovascular disease, neuropathy, and kidney failure \cite{ADATIR, Roglic2016}. Management strategies vary by diabetes type: individuals with Type 1 Diabetes (T1D) rely on daily insulin administration, while those with Type 2 Diabetes (T2D) often begin with lifestyle interventions before progressing to medications or insulin therapy \cite{Roglic2016}. Regardless of the type, effective management hinges on accurate BG monitoring and prediction to enable timely interventions.

Continuous glucose monitoring (CGM) systems have transformed diabetes care by providing real-time glucose readings, offering insights into glucose dynamics that support improved glycemic control and proactive management of hypo- and hyperglycemic events \cite{Heinemann2018, Beck2017, Battelino2019}. However, CGM adoption remains limited due to economic and systemic barriers, particularly among low-income populations, leaving many individuals reliant on self-monitoring of blood glucose (SMBG) methods such as finger prick samples \cite{Oser2021, ADA}. This disparity highlights the need for innovative approaches to make the benefits of CGM-like data accessible to a broader population.

This project addresses this challenge by investigating the potential of sparse blood glucose measurements, such as SMBG, to approximate the predictive power of CGM systems. Using a Subsample-Reconstruct-Analyze (SRA) framework, the CGM data is systematically subsampled to simulate sparse SMBG-like patterns and applied interpolation techniques (linear and cubic spline) to reconstruct glucose profiles. The fidelity of the reconstructed data is evaluated using mean squared error (MSE) and time-in-range metrics, assessing glucose levels across clinically significant thresholds. This project provides insights into the trade-offs between sampling resolution and data fidelity, offering practical implications for diabetes management and sensor optimisation in resource-constrained settings. 

% !TeX encoding = utf8
% !TeX spellcheck = en_US

\section{Related Work}

% Hmm it is not in the scope of our project anymore...

\begin{comment}
    
The use of artificial intelligence to predict blood glucose is not a novel idea. In fact, many have already attempted using different models and strategies to confront this task.

Before venturing into the depths of complex artificial intelligence systems, simple threshold-based algorithmic approaches have been used and have been shown to predict and prevent hypoglycemia with a prediction horizon of 30 minutes but it is thought that artificial intelligence will allow for even more accurate prediction and extended prediction horizons \cite{buckingham_evaluation_2017,duckworth_explainable_2024}.

Currently, various machine learning techniques have been employed which report prediction horizons ranging from 15 minutes to 2 hours. These include artificial neural networks that can be feed-forward or recurrent, support vector machines, genetic programming, random forests and ensemble approaches \cite{woldaregay_data-driven_2019}. The company OneDrop\textsuperscript{\textcircled{\textsc{r}}} even claims a strong prediction accuracy up to 4 hours but little information is available regarding how they implemented their machine learning model, presumably owing to keeping their intellectual property confidential \cite{noauthor_one_nodate}.

Note that using typical machine learning brings about the use of a so-called black-box approach in which the models generate a result without providing an explanation on how it was achieved. In an effort to gather more understanding from these models, some have tried to create explainable tree-based machine learning algorithms to identify the most influential features \cite{duckworth_explainable_2024}. Additionally, hybrid approaches have been developed in which a physiological compartmental model is combined with machine learning \cite{woldaregay_data-driven_2019}. Such compartmental models, which on their own fail to produce clinically relevant results, stem from the early attempts to model the dynamic metabolic processes and present a great opportunity to combine more understandable physiological information into the black-box systems \cite{lehmann_compartmental_1998,mougiakakou_real_2005}. Note that researchers have also looked to use deep learning approaches such as, for example, recurrent neural networks based on long short term memory \cite{deng_deep_2021}.

As can be understood from above, there does unfortunately not yet exist a universal model or approach to forecast blood glucose levels. It appears to still be at the trial-and-error stage in order to find the right method or combination that will yield the best prediction results. More importantly, there is also an absence of a universal way to estimate carbohydrate intake and to account for the effect of certain parameters such as the patient's physical activity, stress and illnesses or infections \cite{woldaregay_data-driven_2019}. Strengthening feature robustness is an important step towards improving the model outcomes.



% New proposition : 

The prediction of blood glucose (BG) levels has been a critical area of research for improving diabetes management. Continuous glucose monitoring (CGM) systems have been widely studied for their ability to provide minute-by-minute data, enabling the development of predictive models for future BG levels. While complex machine learning (ML) techniques such as recurrent neural networks (RNNs) and ensemble models have been employed for long-term predictions, simpler models like linear regression remain valuable for their interpretability and applicability to sparse data scenarios \cite{woldaregay2019data, li2019deep}.

Threshold-based algorithms, which rely on predefined glucose ranges, have been used to predict hypoglycemic and hyperglycemic events for short prediction horizons \cite{buckingham2017evaluation}. However, these methods are less suitable for sparse data inputs, such as those derived from self-monitoring blood glucose (SMBG) measurements.

Hybrid approaches that integrate physiological models with statistical techniques have been investigated to enhance interpretability while maintaining predictive accuracy \cite{lehmann1998compartmental, mougiakakou2005real}. These methods offer insights into glucose dynamics but often require detailed and continuous data, limiting their utility in scenarios where SMBG data is sparse.

Studies focusing on interpolation techniques and sparse data reconstruction have highlighted the potential of simple regression and data-driven methods to approximate CGM trends from SMBG patterns \cite{vig2005smcg}. Interpolation methods, combined with predictive modeling, have shown promise in addressing the challenges posed by irregularly sampled data.

This project builds on these studies by evaluating how subsampling CGM data, simulating SMBG patterns, impacts predictive performance. The findings aim to determine the minimum sampling frequency required to achieve accurate BG predictions, thereby contributing to the broader effort of making predictive tools more accessible for individuals relying on SMBG measurements.


\end{comment}



Subsampling and reconstruction methods are commonly used in time-series analysis to deal with irregular or incomplete data. In diabetes management, most research focuses on predicting glucose levels using CGM data. However, less attention has been given to sparse measurements like those from SMBG to recreate data similar to CGM.

One method, Kalman smoothing, has been used to process glucose data from irregular sources like SMBG or Flash Glucose Monitors (FGM). This method fills in missing data by creating smooth estimates and providing information about how accurate those estimates are. It also corrects errors in the data, such as outliers, making it particularly useful for looking at past glucose trends \cite{Staal2019}.

Another method, a three-level B-spline model, has been used to analyze CGM data and measure glucose variability. This approach helps compare glucose trends between individuals and across different days. It has shown significant differences in glucose variability between groups and is useful for filling gaps in sparse or irregular data \cite{Zheng2011}.

Building on these methods, this project uses a SRA framework to study how reducing the number of glucose readings affects data accuracy. By applying linear and cubic spline interpolation, we explore the balance between fewer data points and the quality of reconstructed glucose profiles.


% !TeX encoding = utf8
% !TeX spellcheck = en_US

\section{Methods}

\subsection{Preprocessing}
The function \texttt{createDataSet} is designed to preprocess the dataset for predictive modeling and analysis of blood glucose monitoring data. It accepts a file path to a CSV dataset and an optional scaler object for normalization, outputting a list of processed data samples (\texttt{carbIntake}) and the scaler used. The function performs several key preprocessing steps to transform raw data into a format suitable for time-series analysis.

The function begins by loading the dataset and applying helper functions to add features related to the timing of fingerstick measurements, meal events, and insulin boluses. It then calculates the rate of change in blood glucose levels over 5-minute and 15-minute intervals, which are included as new features. Irrelevant or redundant columns, such as timestamps and basal insulin levels, are removed to streamline the dataset.

Although optional in this implementation, the function supports data normalization using a \texttt{MinMaxScaler}, ensuring all features fall within a consistent range, typically between 0 and 1. This step facilitates compatibility with machine learning algorithms that are sensitive to the scale of input data.

A critical aspect of the function is its event-based sampling methodology. It identifies events associated with carbohydrate intake and checks for the presence of related events, such as insulin bolus administration or fingerstick measurements, within a 10-minute window. If these conditions are met, the function collects data starting from the earliest event in the window and continues sampling over a 4-hour period, capturing readings every 5 minutes.

The function also generates lagged features for glucose values to provide temporal context. These lagged features capture trends in blood glucose levels by shifting the continuous glucose monitoring data to simulate historical readings. Before finalizing the samples, the function handles missing values by filtering out incomplete rows and replacing any remaining missing values with zero. Unnecessary columns are dropped, and indices are reset to ensure a clean and structured output.

The \texttt{createDataSet} function produces a list of structured data samples that are well-suited for time-series analysis and predictive modeling. This function facilitates the exploration of blood glucose trends in response to external factors, such as meals and insulin dosages, and provides a foundation for forecasting applications.

\subsection{Subsampling and Reconstruction}
The Subsample-Reconstruct-Analyze (SRA) Framework is a computational methodology developed to evaluate the impact of reduced sampling frequencies on continuous glucose monitoring (CGM) data and to assess the effectiveness of reconstruction techniques in maintaining data fidelity. This framework systematically reduces the resolution of CGM data through subsampling, reconstructs the subsampled data using interpolation methods, and analyzes the reconstructed data against the original dataset using statistical and time-based metrics.

\subsubsection{Framework Components}
\paragraph{Subsampling} The subsampling process, implemented in the \texttt{subsample\_df} function, reduces the temporal resolution of the CGM data. By selecting only every \(n\)-th data point, determined by the subsampling rate, the framework mimics scenarios where glucose measurements are taken at lower frequencies due to limitations in data storage, transmission bandwidth, or sensor capabilities. To ensure continuity, the last data point in the original dataset is always included in the subsampled data.

\paragraph{Reconstruction}
Following subsampling, the \texttt{reconstruct\_samples} function reconstructs the original glucose profile by interpolating the missing data points. Two interpolation methods are implemented: linear interpolation (\texttt{interp1d}) and cubic spline interpolation (\texttt{CubicSpline}). The choice of interpolation technique allows flexibility in balancing computational complexity and reconstruction accuracy. The reconstructed data points are generated at evenly spaced intervals, facilitating direct comparisons with the original high-frequency data.

\paragraph{Error Quantification}
The fidelity of the reconstructed data is quantified by the \texttt{create\_statistics} function, which calculates the mean squared error (MSE) between the original and reconstructed datasets. This error metric provides a robust measure of the deviation introduced by subsampling and reconstruction, capturing both the magnitude and consistency of errors across the dataset.

\paragraph{Statistical and Time-in-Range Metrics}
The \texttt{create\_standard\_metrics} function aggregates key metrics for both the original data and reconstructed datasets at various subsampling rates. These metrics include:
\begin{itemize}
	\item \textbf{Mean and Standard Deviation:} Summary statistics capturing the overall distribution of glucose levels.
	\item \textbf{Time-in-Range Metrics:} Proportions of time spent within clinically significant glucose ranges, including:
	\begin{itemize}
		\item Time in range (70–180 mg/dL)
		\item Time in tight range (70–140 mg/dL)
		\item Time in low (<70 mg/dL) and very low (<54 mg/dL) ranges
		\item Time in high (>180 mg/dL) and very high (>250 mg/dL) ranges
	\end{itemize}
\end{itemize}
These metrics are computed for both the original and reconstructed datasets, allowing for a comprehensive evaluation of how subsampling and reconstruction impact the derived statistics.

\paragraph{Visualization}
For a subset of the data, the framework generates visual plots comparing the original glucose profile to its reconstructed counterpart at various subsampling rates. These plots provide an intuitive representation of the effects of subsampling and the performance of the reconstruction methods.

\subsubsection{Use Case and Applications}
The SRA Framework is designed to simulate and analyze scenarios in which glucose monitoring data is recorded at lower frequencies, such as in low-power sensors or during data transmission over constrained networks. It provides insights into the trade-offs between sampling resolution and data fidelity, helping researchers and practitioners understand the limitations and capabilities of CGM systems. By incorporating time-in-range metrics, the framework also ensures that clinically relevant outcomes are preserved under different sampling and reconstruction conditions.

\subsubsection{Advantages of the Framework}
The SRA Framework offers the following advantages:
\begin{itemize}
	\item Flexibility in subsampling rates and interpolation methods
	\item Robust quantification of reconstruction accuracy through MSE
	\item Comprehensive statistical and time-in-range metrics for clinical evaluation
	\item Visualization for intuitive understanding of subsampling effects
\end{itemize}

This framework is particularly relevant for diabetes research and sensor optimization studies, enabling the assessment of data resolution requirements in CGM systems and other time-series analysis domains.



















% !TeX encoding = utf8
% !TeX spellcheck = en_US

\section{Data and Experimental Setup}

\subsection{Data Description}
This study utilizes data collected from 12 individuals with T1D as part of the publicly available OhioT1DM dataset \cite{marling_ohiot1dm_2021}. The dataset contains detailed records of various physiological and behavioral parameters relevant to diabetes management. Key features of the dataset include:

\begin{itemize}
	\item \textbf{Continuous Glucose Monitoring (CGM):} Provides high-frequency glucose measurements, essential for understanding glycemic trends and variability.
	\item \textbf{Self-Monitoring of Blood Glucose (SMBG):} Includes blood glucose readings recorded manually by the participants.
	\item \textbf{Basal Insulin Rate:} Captures the continuously administered insulin through a pump to maintain background glucose control.
	\item \textbf{Bolus Injections:} Records the timing and amount of insulin delivered to counteract meal-related glucose spikes or correct hyperglycemia.
	\item \textbf{Meal Information:} Contains self-reported data on the timing, type of meal, and estimated carbohydrate content consumed by the participants.
	\item \textbf{Other Features:} Includes additional context-specific information such as timestamps and metadata to facilitate comprehensive data analysis.
\end{itemize}

The richness of this dataset allows for detailed exploration and modeling of glucose dynamics, providing opportunities to study how various factors interact to affect glycemic control in individuals with T1D. The inclusion of self-reported data adds a layer of complexity, representing real-world scenarios where variability in estimation and adherence can influence outcomes. This dataset forms the basis for evaluating the performance of the proposed models and methods.

\subsection{Experimental Setup}

%The experiments utilize the OhioT1DM dataset \cite{ohioT1DM_dataset}, which contains minute-by-minute records from 12 individuals with type 1 diabetes (T1D). The dataset includes key variables such as continuous glucose monitoring (CGM) values, self-monitoring blood glucose (SMBG), basal insulin rates, bolus insulin injections, and self-reported meal information.

\subsubsection{Data Preprocessing}
The raw dataset undergoes preprocessing to derive relevant features such as the rate of change in CGM. Additionally, missing values are handled through interpolation and imputation methods to prepare the data for analysis.

\subsubsection{Subsampling and Reconstruction}
To simulate reduced data resolution, the CGM data is subsampled at rates of 1, 2, 4, 8, 16, and 32. Following subsampling, data is reconstructed using cubic spline (\texttt{CubicSpline}) and polynomial of degree five (\texttt{polyfit}) interpolation methods. 

%These methods estimate missing data points, providing reconstructed glucose profiles for further analysis.

\subsubsection{Model Evaluation}
The effectiveness of subsampling and reconstruction is evaluated using:
\begin{itemize}
	\item \textbf{Mean Squared Error (MSE):} Quantifies the difference between original and reconstructed data.
	\item \textbf{Time-in-Range Metrics:} Assesses the proportion of time spent within clinically relevant glucose ranges.
\end{itemize}
Visual inspection is also conducted to compare reconstructed and original data for different subsampling rates.




















% !TeX encoding = utf8
% !TeX spellcheck = en_US

\section{Results}



% !TeX encoding = utf8
% !TeX spellcheck = en_US

\section{Discussion}

Overall, up until a subsampling rate of 8, the reconstructed signal and associated metrics remain relatively coherent with the ground truth. However, as the subsampling rate is increased above 8, the reconstructed signal and associated metrics substantially degrade.

A subsampling rate of 8 corresponds to glucose measurement intervals of 40 minutes. Therefore, from the observations of this study, it would be possible to obtain similar profiles to CGM data from finger prick samples collected every 40 minutes. Daily, assuming a 16 hour awake period and an 8 hour sleep, this would then entail obtaining 24 finger prick samples.

While this is not too absurd of a number, it still remains rather large and can definitely be considered inconvenient and unsustainable for patients. Additionally, this would then also require almost 9 thousand test strips a year, for which one can imagine the cost to still be substantial. Therefore, the results from this study, unfortunately, do not yield a satisfactory enough method to give disadvantaged patients the possibility to reap the benefits of continuous blood glucose monitoring.

Within the context of this study, dynamic and adaptive subsampling could present an additional avenue to explore. The idea would be to obtain finger prick samples at a higher frequency when in proximity to a major glucose level changing event, such as a meal intake, and at smaller frequencies otherwise. 

Further opportunities could lie in the fact that since so many factors such as time of day, meal intake and amount of physical activity, heavily influence glucose levels, these could all be taken into account when reconstructing, or predicting, blood glucose patterns  \cite{noauthor_42_2018}. These could be measured using non-invasive wearable devices and thereby supplement the few finger prick samples that the patient still collects. Incorporating this data would then require straying away from simple interpolation techniques and explore more advanced artificial intelligence and machine learning models.


\begin{comment}
The findings of this study provide critical insights into the trade-offs associated with subsampling rates in continuous glucose monitoring (CGM) data, with implications for both clinical applications and research involving glycemic metrics. This discussion elaborates on the impact of subsampling on key metrics, the fidelity of glucose signal reconstruction, and the clinical relevance of time-in-range metrics.

\subsection{Impact of Subsampling on Mean Blood Glucose and Variability}
The results demonstrate that mean blood glucose values are largely preserved at lower subsampling rates (e.g., rates 1 and 2). However, statistically significant deviations ($p \leq 0.05$) become apparent at higher subsampling rates (e.g., rates 16 and 32). These deviations highlight interpatient variability in mean glucose metrics when subsampling. This variability likely reflects differences in the temporal patterns of glucose fluctuations, emphasizing the need for individualized considerations when applying data reduction techniques.
Overall, these results suggest that while subsampling minimally affects mean blood glucose values at lower rates, higher subsampling rates may introduce significant deviations, particularly for specific patients.

Furthermore, the observed reduction in standard deviation at higher subsampling rates is particularly noteworthy. While standard deviation—a critical metric for evaluating glucose variability(CITATION NEEDED) —remains stable at lower rates (e.g., rates 1, 2, and 4), substantial reductions are observed at rates of 16 and 32. This trend indicates that extreme subsampling may suppress variability, potentially obscuring clinically meaningful fluctuations in glucose levels. Given the importance of variability metrics (CITATION NEEDED) for assessing glycemic control and predicting adverse outcomes, the findings underscore the limitations of aggressive subsampling in preserving the integrity of these metrics.
% These findings emphasize the trade-offs between data reduction and the preservation of variability in glucose measurements, with implications for accurately assessing glycemic control and variability metrics.

\subsection{Reconstruction Error and Signal Fidelity}
The mean squared error (MSE) analysis, as depicted in Figure~\ref{fig:mses}, reveals that reconstruction fidelity is highly dependent on the subsampling rate. At lower rates (e.g., 1, 2, and 4), MSE values remain low, indicating minimal information loss. However, as the subsampling rate increases to 16 and 32, the MSE rises sharply, reflecting substantial degradation in reconstruction accuracy. This loss of fidelity is further evidenced by the distortion of reconstructed glucose signals at higher rates, as shown in Figure~\ref{fig:SR_combined}. Key features of glucose dynamics, such as peak amplitudes and nadirs, are progressively smoothed or distorted, undermining the ability to accurately capture the temporal complexity of glucose fluctuations.
These results emphasize the trade-off between sampling frequency and reconstruction fidelity, with higher subsampling rates introducing substantial error and variability in capturing blood glucose dynamics.

The comparative performance of cubic spline and degree-5 polynomial interpolation methods highlights the superiority of the former for reconstructing glucose signals, particularly at lower subsampling rates. Although the performance gap narrows at higher rates, the consistently lower MSE values for cubic splines support its use as the preferred method for CGM data reconstruction. The widening variability in MSEs at higher subsampling rates also underscores the heterogeneity in reconstruction accuracy across patients, suggesting that some individuals may be more susceptible to information loss under data reduction.

\subsection{Glycemic Ranges and Clinical Implications}
The analysis of time spent in different glycemic ranges further elucidates the impact of subsampling on clinically relevant metrics. For most patients, the time spent in hypoglycemia (\textless 70~mg/dL) and the target range (70--180~mg/dL) remains stable across lower subsampling rates (e.g., 1, 2, and 4). However, deviations emerge at higher rates, particularly for patients with greater glycemic variability or those spending significant time in extreme glucose ranges.

The reduction in the time spent in hyperglycemia (\textgreater 180~mg/dL) at higher subsampling rates is particularly concerning, as it may lead to an underestimation of prolonged periods of elevated glucose levels. This underestimation is most pronounced in patients with a higher initial percentage of time in hyperglycemia, as observed for specific patients (e.g., orange and blue dots in Figure~\ref{fig:time_in_ranges}). Such distortions could have significant implications for clinical decision-making, as they may compromise the accuracy of assessments regarding glycemic control and treatment efficacy (CITATION NEEDED).

\subsection{Balancing Data Reduction and Analytical Integrity}
These findings emphasize the trade-offs between data reduction and the preservation of clinically meaningful metrics in CGM data. While subsampling at lower rates (e.g., 1, 2, and 4) appears to maintain the integrity of mean values, variability metrics, and time-in-range distributions, higher rates introduce substantial reconstruction errors and distort key glycemic metrics. Thus, the selection of subsampling rates should be guided by the specific analytical objectives and the need to balance computational efficiency with the preservation of data fidelity.

\subsection{Real-World Constraints and Practical Considerations}
The findings of this study highlight the potential trade-offs between data fidelity and practical constraints in continuous glucose monitoring (CGM). In real-world scenarios, achieving the balance between data accuracy and the limitations of resource availability is critical. High-resolution data collection requires significant storage capacity and computational resources, both of which may pose challenges in resource-constrained environments or in long-term monitoring applications. For instance, patients using CGM devices for months or years may experience difficulties with data storage or retrieval due to the sheer volume of high-frequency recordings.

Moreover, the transmission of high-resolution CGM data, particularly in real-time, may face challenges such as limited bandwidth in telemedicine applications or power constraints in portable devices (CITATION NEEDED). Subsampling techniques could provide a practical solution by reducing the data volume while retaining key trends and patterns, thereby enabling more efficient data transmission and storage. However, the results of this study emphasize that care must be taken to select appropriate subsampling rates to ensure that critical information, particularly related to glycemic variability, is not lost.

These constraints are particularly relevant for remote or resource-limited settings, where access to high-quality monitoring devices and infrastructure may be limited (CITATION NEEDED). Subsampling approaches, when optimized, could make CGM technology more accessible and affordable in such settings (CITATION NEEDED). However, it is important to note that the trade-offs introduced by subsampling must not compromise the accuracy of clinically relevant metrics, such as time in range, or the detection of rapid glucose fluctuations.

Future work could explore adaptive subsampling techniques that dynamically adjust sampling rates based on observed glycemic variability or patient-specific characteristics. Such approaches may offer a more tailored balance between minimizing data volume and preserving the integrity of critical glucose measurements. Additionally, advancements in data compression and signal processing algorithms could further alleviate the constraints associated with high-frequency data collection, offering alternative pathways for overcoming the limitations observed in this study.

\subsection{Limitations and Future Work}
\subsubsection{Limitations}
While this study provides valuable insights into the impact of subsampling on blood glucose monitoring, several limitations warrant consideration. First, the analysis is based on a specific dataset and may not fully capture the diversity of glycemic patterns observed across different populations, such as pediatric patients or individuals with distinct metabolic disorders. The observed variability in subsampling effects across patients underscores the importance of validating these findings across larger and more diverse cohorts.

Second, this study assumes consistent device performance and signal quality at all sampling rates. In real-world CGM systems, noise, calibration inaccuracies, and sensor drift can influence the accuracy of both raw and subsampled data. These factors, not addressed explicitly in this analysis, may amplify the errors observed at higher subsampling rates. Future studies should incorporate realistic noise models to evaluate the robustness of subsampling techniques under variable signal quality.

Additionally, the use of cubic spline interpolation for data reconstruction, while effective, represents only one of many possible methods. The performance of alternative interpolation techniques, such as machine learning-based approaches or adaptive filtering methods, remains unexplored and could yield different outcomes. Furthermore, the current study evaluates subsampling based on fixed rates, which may not fully reflect real-world scenarios where glycemic variability and clinical context could influence sampling needs.

\subsubsection{Future Work}
Building on the findings of this study, future research could explore several promising directions:

\begin{itemize}
	\item \textbf{Adaptive Subsampling Strategies:} One key area of interest is the development of adaptive subsampling techniques that dynamically adjust sampling rates based on glycemic variability or patient-specific factors. Such strategies could optimize data acquisition by focusing on critical periods, such as times of rapid glucose fluctuations, while reducing sampling during stable periods.
	
	% \item \textbf{Noise and Signal Quality Modeling:} Incorporating realistic models of sensor noise and device errors would provide a more comprehensive evaluation of subsampling techniques under real-world conditions. This could help determine the thresholds at which subsampling becomes clinically unacceptable.
	
	\item \textbf{Alternative Reconstruction Methods:} Exploring advanced interpolation and reconstruction techniques, such as machine learning algorithms or hybrid methods, may improve the accuracy of data reconstruction, particularly at higher subsampling rates. Comparing the performance of these methods against traditional techniques like cubic splines could yield insights into their practical applicability.
	
	\item \textbf{Clinical Outcomes:} While this study focuses on the accuracy of reconstructed glucose data, future work could assess the clinical implications of subsampling on decision-making and patient outcomes. For example, the impact of subsampling on insulin dosing recommendations, hypoglycemia prediction, and time-in-range metrics should be systematically evaluated.
	
	\item \textbf{Real-Time Applications:} In scenarios where data transmission bandwidth or device power is limited, subsampling could be implemented in real-time. Investigating the feasibility and effectiveness of real-time adaptive subsampling in CGM devices could bridge the gap between theoretical findings and practical applications.
	
	\item \textbf{Longitudinal Studies:} Long-term studies are needed to assess the impact of subsampling on extended glucose monitoring periods. These studies should examine the trade-offs between reduced data volume and accuracy over weeks or months of continuous monitoring, providing insights into the feasibility of subsampling in chronic disease management.
	
	\item \textbf{Personalized Subsampling:} Future research could investigate personalized subsampling strategies tailored to individual glycemic patterns and lifestyle factors. Such approaches could enhance the usability and accuracy of CGM systems by accounting for patient-specific variability in glucose dynamics.
\end{itemize}

\end{comment}
% !TeX encoding = utf8
% !TeX spellcheck = en_US

\section{Conclusion}

%In summary, while this study demonstrates the potential trade-offs associated with subsampling, addressing the identified limitations and pursuing the outlined future directions could significantly enhance the applicability and reliability of subsampling techniques in blood glucose monitoring. These efforts will be essential to ensure that CGM systems continue to provide high-quality data while meeting the practical constraints of real-world applications.
 
 This study artificially generated sparse glucose measurement data by subsampling, at different rates, continuous glucose monitoring data obtained from patients with type I diabetes. This study investigated whether simple interpolation techniques, such as cubic spline and polynomial of degree five, were capable of reliably reconstructing and approximating the original signal. While achievable at an interval of 40 minutes, this timeframe is still too short to allow finger prick sample data to compare to that of continuous glucose monitoring.
 
% !TeX encoding = utf8
% !TeX spellcheck = en_US

\section{Acknowledgment}
We would like to thank Maria Panagiotou and Knut Strommen for guiding us throughout this project and for giving us a pleasant and instructive experience.


%----------------------------------------------------------------------------------------
%	 REFERENCES
%----------------------------------------------------------------------------------------

\printbibliography % Output the bibliography

%----------------------------------------------------------------------------------------

\end{document}
