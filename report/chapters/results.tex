% !TeX encoding = utf8
% !TeX spellcheck = en_US

\section{Results}

\begin{table*}
	\centering
	\begin{tabular}{l |cc|cc|cc|}
		\hline
		& \multicolumn{2}{c|}{Parkinson patient} & \multicolumn{2}{c|}{Participant A} & \multicolumn{2}{c|}{Participant B} \\
		\cmidrule(r){2-7}
		& \multicolumn{1}{c}{\textbf{Left foot}} & \multicolumn{1}{c|}{\textbf{Right foot}} & \multicolumn{1}{c}{\textbf{Left foot}} & \multicolumn{1}{c|}{\textbf{Right foot}} & \multicolumn{1}{c}{\textbf{Left foot}} & \multicolumn{1}{c|}{\textbf{Right foot}} \\
		\hline
		Counted total steps & \multicolumn{2}{c|}{n/a} & \multicolumn{2}{c|}{517} & \multicolumn{2}{c|}{505} \\
		Number of steps & 18 & 18 & 259 & 261& 254 & 254\\
		Total steps & \multicolumn{2}{c|}{36} & \multicolumn{2}{c|}{520 [99.4\%]} & \multicolumn{2}{c|}{508 [99.4\%]} \\
		Mean swing time (s) & 0.072 & 0.073 & 0.105 & 0.113 & 0.113 & 0.113 \\
		Mean stance time (s) & 1.151 & 1.151 & 1.077 & 1.069 & 1.041 & 1.041 \\
		Mean stride time (s) & 1.229 & 1.230 & 1.183 & 1.183 & 1.155 & 1.155 \\
		Mean cadence (steps/min) & 97.60 & 97.51 & 101.45 & 101.39 & 103.89 & 103.83\\
		\hline
	\end{tabular}
	\caption{Summary of the number of steps and additionally calculated gait parameters.}
	\label{tab:results}
\end{table*}
The results obtained from both participants alongside the Parkinson patient's results for comparison can be found summarized in Table \ref{tab:results}.

For Participant A, the accelerometer detected 259 steps for the left foot and 261 steps for the right foot, for a total of 520 steps. For this participant, physically counting the steps registered, by average, 517 steps. Using Equation \eqref{eqn:accuracyEquation}, this leads to a 99.4\% accuracy in the ability of the accelerometer to count steps.

For Participant B, the accelerometer detected 254 steps for both feet, resulting in a total of 508 steps. For this participant, physically counting the steps registered, by average, 505 steps. Similarly as for participant A, this leads to a 99.4\% accuracy in the ability of the accelerometer to count steps.

For the Parkinson's disease patient, no recorded number of counted total steps was available. However, on the segment of data analyzed, 36 total steps were counted, with 18 steps for each foot.

Moving onto the temporal gait parameters. For Participant A, the mean swing time was 0.105 seconds for the left foot and 0.113 seconds for the right foot. The mean stance time was 1.077 seconds for the left foot and 1.069 seconds for the right foot. The mean stride time was 1.183 seconds for both feet.

For Participant B, the mean swing time was 0.113 seconds for the left foot and 0.113 seconds for the right foot. The mean stance time was 1.041 seconds for the left foot and 1.041 seconds for the right foot. The mean stride time was 1.155 seconds for both feet.

For the Parkinson's disease patient, the mean swing time was 0.072 seconds for the left foot and 0.073 seconds for the right foot. The mean stance time was 1.151 seconds for both feet. The mean stride time was 1.229 seconds for the left foot and 1.230 seconds for the right foot.

Both participants have a mean cadence greater than 100 steps per minute as the mean cadence for participant A was 101.46 steps per minute for the left foot and 101.39 steps per minute for the right foot. Participant B had a slightly faster mean cadence with 103.89 steps per minute for the left foot and 103.83 steps per minute for the right foot.

In contrast, the Parkinson's disease patient had a mean cadence of less than 100 steps per minute as the mean cadence was 97.60 steps per minute for the left foot and 97.51 steps per minute for the right foot.


