% !TeX encoding = utf8
% !TeX spellcheck = en_US

\section{Discussion}
Given the very high accuracy of the calculated step count using the accelerometers compared to the actual counted step count, this would corroborate the use of accelerometers as a simple step counter. 

However, there are a few limitations that are needed to be taken into account. In this experiment, the participants walked at a regular pace which lead to a consistent acceleration signal. Additionally, given the large curvature radius of the track, the participants walked approximately in a straight line, meaning that they never changed directions. Essentially, this experiment constrained the walking to a very simple scenario, hence, the subsequent data analysis was performed on this type of data. Further research would be required to evaluate and expand the use of accelerometers in a more complex walking situation. 

Additionally, these findings reveal a difference in the temporal gait parameters between healthy individuals and a Parkinson's disease patient. Comparing with the differences found by the meta-analysis performed by Zanardi and colleagues, this study also observed reduced swing time and increased stance time in the Parkinson's disease patient compared to healthy participants. Note that Zanardi and colleagues report single and double support time separately and that stance time comprises both of those measures. However, Zanardi and colleagues did report an increase in cadence for Parkinson's diseased patients whereas this study observed the opposite, notably, a decreased cadence \cite{zanardi_gait_2021}.

A limitation to take into account was that the accelerometer data from the healthy participants and the Parkinson's disease patient were acquired in different settings which means there may be confounding variables that affect the temporal gait parameters that have not been taken into account in this study. 

%Does not fit with idea as a whole 
%In addition, there is a high likelihood that the healthy participants in this study are much younger than the Parkinson's disease patient which means that normal age-related differences in gait could also explain the observed differences in gait in this study. Further, age-matched research would be required to confirm

Furthermore, it is not possible to declare anything more paramount from these findings other than there are some observed differences in gait between healthy individuals and a patient suffering from Parkinson's disease. This study was limited to only a few temporal gait parameters while gait can be characterized by many more, including spatial parameters. Therefore, to undoubtedly make the distinction between healthy and Parkinson's disease gait, all these auxiliary parameters should be regarded and further research is required to determine the significance of these changes.

%These findings underscore the distinct gait abnormalities associated with Parkinson's disease, characterized by slower and less rhythmic movements, further highlighting the potential utility of gait analysis in clinical assessment and monitoring of Parkinson's disease progression . 
