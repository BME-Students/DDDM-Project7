% !TeX encoding = utf8
% !TeX spellcheck = en_US

\section{Discussion}

Overall, up until a subsampling rate of 8, the reconstructed signal and associated metrics remain relatively coherent with the ground truth. However, as the subsampling rate is increased above 8, the reconstructed signal and associated metrics substantially degrade.

A subsampling rate of 8 corresponds to glucose measurement intervals of 40 minutes. Therefore, from the observations of this study, it would be possible to obtain similar profiles to CGM data from finger prick samples collected every 40 minutes. Daily, assuming a 16 hour awake period and an 8 hour sleep, this would then entail obtaining 24 finger prick samples.

While this is not too absurd of a number, it still remains rather large and can definitely be considered inconvenient and unsustainable for patients. Additionally, this would then also require almost 9 thousand test strips a year, for which one can imagine the cost to still be substantial. Therefore, the results from this study, unfortunately, do not yield a satisfactory enough method to give disadvantaged patients the possibility to reap the benefits of continuous blood glucose monitoring.

Within the context of this study, dynamic and adaptive subsampling could present an additional avenue to explore. The idea would be to obtain finger prick samples at a higher frequency when in proximity to a major glucose level changing event, such as a meal intake, and at smaller frequencies otherwise. 

Further opportunities could lie in the fact that since so many factors such as time of day, meal intake and amount of physical activity, heavily influence glucose levels, these could all be taken into account when reconstructing, or predicting, blood glucose patterns  \cite{noauthor_42_2018}. These could be measured using non-invasive wearable devices and thereby supplement the few finger prick samples that the patient still collects. Incorporating this data would then require straying away from simple interpolation techniques and explore more advanced artificial intelligence and machine learning models.


\begin{comment}
The findings of this study provide critical insights into the trade-offs associated with subsampling rates in continuous glucose monitoring (CGM) data, with implications for both clinical applications and research involving glycemic metrics. This discussion elaborates on the impact of subsampling on key metrics, the fidelity of glucose signal reconstruction, and the clinical relevance of time-in-range metrics.

\subsection{Impact of Subsampling on Mean Blood Glucose and Variability}
The results demonstrate that mean blood glucose values are largely preserved at lower subsampling rates (e.g., rates 1 and 2). However, statistically significant deviations ($p \leq 0.05$) become apparent at higher subsampling rates (e.g., rates 16 and 32). These deviations highlight interpatient variability in mean glucose metrics when subsampling. This variability likely reflects differences in the temporal patterns of glucose fluctuations, emphasizing the need for individualized considerations when applying data reduction techniques.
Overall, these results suggest that while subsampling minimally affects mean blood glucose values at lower rates, higher subsampling rates may introduce significant deviations, particularly for specific patients.

Furthermore, the observed reduction in standard deviation at higher subsampling rates is particularly noteworthy. While standard deviation—a critical metric for evaluating glucose variability(CITATION NEEDED) —remains stable at lower rates (e.g., rates 1, 2, and 4), substantial reductions are observed at rates of 16 and 32. This trend indicates that extreme subsampling may suppress variability, potentially obscuring clinically meaningful fluctuations in glucose levels. Given the importance of variability metrics (CITATION NEEDED) for assessing glycemic control and predicting adverse outcomes, the findings underscore the limitations of aggressive subsampling in preserving the integrity of these metrics.
% These findings emphasize the trade-offs between data reduction and the preservation of variability in glucose measurements, with implications for accurately assessing glycemic control and variability metrics.

\subsection{Reconstruction Error and Signal Fidelity}
The mean squared error (MSE) analysis, as depicted in Figure~\ref{fig:mses}, reveals that reconstruction fidelity is highly dependent on the subsampling rate. At lower rates (e.g., 1, 2, and 4), MSE values remain low, indicating minimal information loss. However, as the subsampling rate increases to 16 and 32, the MSE rises sharply, reflecting substantial degradation in reconstruction accuracy. This loss of fidelity is further evidenced by the distortion of reconstructed glucose signals at higher rates, as shown in Figure~\ref{fig:SR_combined}. Key features of glucose dynamics, such as peak amplitudes and nadirs, are progressively smoothed or distorted, undermining the ability to accurately capture the temporal complexity of glucose fluctuations.
These results emphasize the trade-off between sampling frequency and reconstruction fidelity, with higher subsampling rates introducing substantial error and variability in capturing blood glucose dynamics.

The comparative performance of cubic spline and degree-5 polynomial interpolation methods highlights the superiority of the former for reconstructing glucose signals, particularly at lower subsampling rates. Although the performance gap narrows at higher rates, the consistently lower MSE values for cubic splines support its use as the preferred method for CGM data reconstruction. The widening variability in MSEs at higher subsampling rates also underscores the heterogeneity in reconstruction accuracy across patients, suggesting that some individuals may be more susceptible to information loss under data reduction.

\subsection{Glycemic Ranges and Clinical Implications}
The analysis of time spent in different glycemic ranges further elucidates the impact of subsampling on clinically relevant metrics. For most patients, the time spent in hypoglycemia (\textless 70~mg/dL) and the target range (70--180~mg/dL) remains stable across lower subsampling rates (e.g., 1, 2, and 4). However, deviations emerge at higher rates, particularly for patients with greater glycemic variability or those spending significant time in extreme glucose ranges.

The reduction in the time spent in hyperglycemia (\textgreater 180~mg/dL) at higher subsampling rates is particularly concerning, as it may lead to an underestimation of prolonged periods of elevated glucose levels. This underestimation is most pronounced in patients with a higher initial percentage of time in hyperglycemia, as observed for specific patients (e.g., orange and blue dots in Figure~\ref{fig:time_in_ranges}). Such distortions could have significant implications for clinical decision-making, as they may compromise the accuracy of assessments regarding glycemic control and treatment efficacy (CITATION NEEDED).

\subsection{Balancing Data Reduction and Analytical Integrity}
These findings emphasize the trade-offs between data reduction and the preservation of clinically meaningful metrics in CGM data. While subsampling at lower rates (e.g., 1, 2, and 4) appears to maintain the integrity of mean values, variability metrics, and time-in-range distributions, higher rates introduce substantial reconstruction errors and distort key glycemic metrics. Thus, the selection of subsampling rates should be guided by the specific analytical objectives and the need to balance computational efficiency with the preservation of data fidelity.

\subsection{Real-World Constraints and Practical Considerations}
The findings of this study highlight the potential trade-offs between data fidelity and practical constraints in continuous glucose monitoring (CGM). In real-world scenarios, achieving the balance between data accuracy and the limitations of resource availability is critical. High-resolution data collection requires significant storage capacity and computational resources, both of which may pose challenges in resource-constrained environments or in long-term monitoring applications. For instance, patients using CGM devices for months or years may experience difficulties with data storage or retrieval due to the sheer volume of high-frequency recordings.

Moreover, the transmission of high-resolution CGM data, particularly in real-time, may face challenges such as limited bandwidth in telemedicine applications or power constraints in portable devices (CITATION NEEDED). Subsampling techniques could provide a practical solution by reducing the data volume while retaining key trends and patterns, thereby enabling more efficient data transmission and storage. However, the results of this study emphasize that care must be taken to select appropriate subsampling rates to ensure that critical information, particularly related to glycemic variability, is not lost.

These constraints are particularly relevant for remote or resource-limited settings, where access to high-quality monitoring devices and infrastructure may be limited (CITATION NEEDED). Subsampling approaches, when optimized, could make CGM technology more accessible and affordable in such settings (CITATION NEEDED). However, it is important to note that the trade-offs introduced by subsampling must not compromise the accuracy of clinically relevant metrics, such as time in range, or the detection of rapid glucose fluctuations.

Future work could explore adaptive subsampling techniques that dynamically adjust sampling rates based on observed glycemic variability or patient-specific characteristics. Such approaches may offer a more tailored balance between minimizing data volume and preserving the integrity of critical glucose measurements. Additionally, advancements in data compression and signal processing algorithms could further alleviate the constraints associated with high-frequency data collection, offering alternative pathways for overcoming the limitations observed in this study.

\subsection{Limitations and Future Work}
\subsubsection{Limitations}
While this study provides valuable insights into the impact of subsampling on blood glucose monitoring, several limitations warrant consideration. First, the analysis is based on a specific dataset and may not fully capture the diversity of glycemic patterns observed across different populations, such as pediatric patients or individuals with distinct metabolic disorders. The observed variability in subsampling effects across patients underscores the importance of validating these findings across larger and more diverse cohorts.

Second, this study assumes consistent device performance and signal quality at all sampling rates. In real-world CGM systems, noise, calibration inaccuracies, and sensor drift can influence the accuracy of both raw and subsampled data. These factors, not addressed explicitly in this analysis, may amplify the errors observed at higher subsampling rates. Future studies should incorporate realistic noise models to evaluate the robustness of subsampling techniques under variable signal quality.

Additionally, the use of cubic spline interpolation for data reconstruction, while effective, represents only one of many possible methods. The performance of alternative interpolation techniques, such as machine learning-based approaches or adaptive filtering methods, remains unexplored and could yield different outcomes. Furthermore, the current study evaluates subsampling based on fixed rates, which may not fully reflect real-world scenarios where glycemic variability and clinical context could influence sampling needs.

\subsubsection{Future Work}
Building on the findings of this study, future research could explore several promising directions:

\begin{itemize}
	\item \textbf{Adaptive Subsampling Strategies:} One key area of interest is the development of adaptive subsampling techniques that dynamically adjust sampling rates based on glycemic variability or patient-specific factors. Such strategies could optimize data acquisition by focusing on critical periods, such as times of rapid glucose fluctuations, while reducing sampling during stable periods.
	
	% \item \textbf{Noise and Signal Quality Modeling:} Incorporating realistic models of sensor noise and device errors would provide a more comprehensive evaluation of subsampling techniques under real-world conditions. This could help determine the thresholds at which subsampling becomes clinically unacceptable.
	
	\item \textbf{Alternative Reconstruction Methods:} Exploring advanced interpolation and reconstruction techniques, such as machine learning algorithms or hybrid methods, may improve the accuracy of data reconstruction, particularly at higher subsampling rates. Comparing the performance of these methods against traditional techniques like cubic splines could yield insights into their practical applicability.
	
	\item \textbf{Clinical Outcomes:} While this study focuses on the accuracy of reconstructed glucose data, future work could assess the clinical implications of subsampling on decision-making and patient outcomes. For example, the impact of subsampling on insulin dosing recommendations, hypoglycemia prediction, and time-in-range metrics should be systematically evaluated.
	
	\item \textbf{Real-Time Applications:} In scenarios where data transmission bandwidth or device power is limited, subsampling could be implemented in real-time. Investigating the feasibility and effectiveness of real-time adaptive subsampling in CGM devices could bridge the gap between theoretical findings and practical applications.
	
	\item \textbf{Longitudinal Studies:} Long-term studies are needed to assess the impact of subsampling on extended glucose monitoring periods. These studies should examine the trade-offs between reduced data volume and accuracy over weeks or months of continuous monitoring, providing insights into the feasibility of subsampling in chronic disease management.
	
	\item \textbf{Personalized Subsampling:} Future research could investigate personalized subsampling strategies tailored to individual glycemic patterns and lifestyle factors. Such approaches could enhance the usability and accuracy of CGM systems by accounting for patient-specific variability in glucose dynamics.
\end{itemize}

\end{comment}