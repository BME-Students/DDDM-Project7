% !TeX encoding = utf8
% !TeX spellcheck = en_US

\section{Conclusion}
 This study validates the use of accelerometers as a simple step counter in a controlled environment but additional research is required to validate their use in more complex scenarios.
 
 This study reveals that there are differences measured using accelerometers in temporal gait parameters observed in a Parkinson's disease patient compared to healthy individuals. However, significant further research is required to establish the full picture of and to make sense of healthy versus diseased gait characteristics. This further research has the potential to employ those future findings in a clinical setting. 
 
 Hopefully, this study has been a convincing look at accelerometers as a strong tool in assessing gait characteristics and a motivation for their continued use in further research with the goal of improving outcomes in patient health and enhancing the quality of life for patients affected by Parkinson's disease.
 