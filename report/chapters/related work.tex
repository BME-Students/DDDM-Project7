% !TeX encoding = utf8
% !TeX spellcheck = en_US

\section{Related Work}

Subsampling and reconstruction methods are commonly used in time-series analysis to deal with irregular or incomplete data. In diabetes management, most research focuses on predicting glucose levels using CGM data. However, less attention has been given to sparse measurements like those from SMBG to recreate data similar to CGM.

One method, Kalman smoothing, has been used to process glucose data from irregular sources like SMBG or Flash Glucose Monitors (FGM). This method fills in missing data by creating smooth estimates and providing information about how accurate those estimates are. It also corrects errors in the data, such as outliers, making it particularly useful for looking at past glucose trends \cite{Staal2019}.

Another method, a three-level B-spline model, has been used to analyze CGM data and measure glucose variability. This approach helps compare glucose trends between individuals and across different days. It has shown significant differences in glucose variability between groups and is useful for filling gaps in sparse or irregular data \cite{Zheng2011}.

Building on these methods, this project uses a SRA framework to study how reducing the number of glucose readings affects data accuracy. By applying cubic spline and polynomial interpolation, we explore the balance between fewer data points and the quality of reconstructed glucose profiles.

