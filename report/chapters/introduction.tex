% !TeX encoding = utf8
% !TeX spellcheck = en_US

\section{Introduction}

$\text{C}_6\text{H}_{12}\text{O}_6$, a molecule known as glucose, is the product of the digestive breakdown of carbohydrates, which alongside lipids and proteins, fuel the metabolic processes of the human body \cite{noauthor_human_nodate}. Glucose is distributed throughout the body and delivered to the various tissues and cells via the bloodstream while the pancreas, by using the hormones insulin and glucagon, carefully works to maintain a healthy glucose concentration in the blood \cite{roder_pancreatic_2016}.

Diabetes, categorized as either type I or type II, is a chronic endocrine disease in which the body can no longer control the proper homeostasis of blood glucose concentration, leading to hypo- and hyperglycemia. Type I diabetes occurs when the pancreas produces insufficient or even no insulin at all. In contrast, type II diabetes is marked by the body's resistance to and inability to effectively use insulin \cite{world_health_organization_global_2016}. Recent estimates reveal that roughly 529 million people worldwide are suffering from diabetes, highlighting the magnitude of this disease \cite{ong_global_2023}.

Health complications, morbidity and mortality are significant risks plaguing individuals living with diabetes. These include, for example, loss of vision, nerve damage, end-stage renal disease, higher rates of cardiovascular events such as stroke and myocardial infarction, increased rate of cancer, increased rates of physical and cognitive disability and premature death \cite{world_health_organization_global_2016}. Hence, it is imperative for diabetics to keep their blood glucose levels within the healthy range, considered to be 70 to 180 mg/dL \cite{noauthor_time_2021}.

%Old sentence
%Fortunately, individuals are able to live healthy lives and reduce their number of complications through proper and robust management.

Proper and robust management, allowing for this glycemic control, ensures that patients can minimize complications and lead healthy lives. Being distinct facets of diabetes, type I and type II have slightly heterogeneous management strategies. For type I, patients require daily administration of exogenous insulin, with the dosage adjusted according to carbohydrate intake and exercise, to mimic the insulin secretion pattern in the absence of disease. For type II, the treatment regiment is more case-dependent, as initial consultation may focus on lifestyle changes for weight-reduction while second line therapies include drugs and oral medications to reduce hepatic glucose production or to reduce insulin resistance \cite{bilous_handbook_2021}. Insulin therapy, as described for type I patients, may also be a treatment option for type II patients \cite{noauthor_patient_nodate}. 

%Old sentence
%In the specific case of blood glucose, on top of monitoring the levels, they need to interpret such complex patterns . 

To this end, diabetes is a disease that requires the patient to take an active role, a process referred to as self-management. This brings about additional burden and emotional distress which can result in sub-optimal management \cite{adu_enablers_2019}. In the particular instance of blood glucose, not only do patients need to monitor their levels but they also need to continuously understand and interpret their patterns, a task that is already challenging enough for experienced clinicians given the shear amount of data and its seeming stochasticity. Therefore, it is clear that new assistive and sophisticated tools are required, and with the current advances in the field, artificial intelligence presents itself as a leading candidate \cite{mayya_need_2024}.

Artificial intelligence can be used to support many aspects of diabetes care such as assessing risk of developing the disease, diagnosis, lifestyle recommendations among others \cite{mayya_need_2024}. Of interest in this study is the use of artificial intelligence, and so called data-driven methods, in the prediction of blood glucose levels. This subsequently raises the question, given if it is possible to predict future glucose values, what is the prediction horizon, or how far into the future can this prediction still be accurate. Addressing this question is impactful as it could allow for patients using only single finger sticks to have an idea of what their continuous glucose would be or could notify patients using continuous glucose monitoring that they need to take corrective action to avoid an adverse glycemic event.











