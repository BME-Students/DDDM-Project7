% !TeX encoding = utf8
% !TeX spellcheck = en_US

\section{Introduction}

Diabetes, categorized as either type I (T1D) or type II (T2D), is a chronic endocrine disease in which the body can no longer control the proper homeostasis of blood glucose (BG) concentration, leading to hypo-  and hyper-glycemia. T1D occurs when the pancreas produces insufficient or even no insulin at all. In contrast, T2D is marked by the body's resistance to and inability to effectively use insulin \cite{world_health_organization_global_2016}. Maintaining BG levels within a healthy range (70--180~mg/dL) is critical for individuals with diabetes to prevent complications such as cardiovascular disease, neuropathy, and kidney failure \cite{ADATIR, Roglic2016}. 

Proper and robust management, allowing for this glycemic control, ensures that patients can lead healthy lives. Management strategies vary by disease etiology: individuals with T1D rely on daily insulin administration, while those with T2D often begin with lifestyle interventions before progressing to medications or insulin therapy \cite{Roglic2016}. Broadly speaking, effective management hinges on accurate BG monitoring and prediction to enable timely interventions.

Continuous glucose monitoring (CGM) systems have transformed diabetes care by providing real-time glucose readings, offering insights into glucose dynamics that support improved glycemic control and proactive management of hypo- and hyperglycemic events \cite{Heinemann2018, Beck2017, Battelino2019}. However, CGM adoption remains limited due to economic and systemic barriers, particularly among low-income populations, leaving many individuals reliant on self-monitoring of blood glucose (SMBG) methods such as fingerstick samples \cite{Oser2021, ADA}. This disparity highlights the need for innovative approaches to make the benefits of CGM-like data accessible to a broader population.

This project addresses this challenge by investigating the potential of sparse blood glucose measurements, such as SMBG, to approximate the predictive power of CGM systems. Using a Subsample-Reconstruct-Analyze (SRA) framework, the CGM data is subsampled to simulate sparse SMBG-like patterns, and interpolation techniques are applied to reconstruct the glucose profiles. The fidelity of the reconstructed data is evaluated using mean squared error (MSE) and time-in-range metrics. 

%This project provides insights into the trade-offs between sampling resolution and data fidelity, offering practical implications for diabetes management and sensor optimisation in resource-constrained settings. 
