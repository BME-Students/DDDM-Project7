% !TeX encoding = utf8
% !TeX spellcheck = en_US

\section{Introduction}

As obvious it may seem, walking governs our everyday life. It is often taken for granted but deeper insight through instrumented measurements can prove fruitful. While the origin of human bipedality may remain a mystery of evolution, our increasing study of this innate biomechanical ability reveals rather important outcomes in the context of medicine. 

The pattern or manner of walking, also known as ambulation and locomotion, is called gait. At surface level, gait would strike as rudimentary since infants learn to walk very early on. Therefore, it may come as a surprise that, under the hood, human gait is in fact quite a complex phenomenon. During the gait cycle, the central and peripheral nervous systems are called into action to coordinate the movement of the musculoskeletal system. This then leads to an intricate interaction between the pelvis, hips, knees and ankles of the lower body but additionally invokes the motion of the trunk, arms and head of the upper body \cite{webster_principles_2019}.

Gait is influenced by many factors: physical, cognitive, sensory, cardiovascular and metabolic. Therefore, since a change in any of these factors directly alters gait, the latter is said to serve as an integrative measure of health \cite{lebrasseur_gait_2019}. Hence, gait analysis can be a useful tool for studying various diseases, such as, for example, Parkinson's disease.

Parkinson's disease is a basal ganglia disorder in which the dopaminergic neurons of the substantia nigra degenerate and is characterized by hypokinesia, decreased motor activity. The basal ganglia are structures of the cerebrum that control movement. Since Parkinson's disease occurs most often after the age of 60, it is known as a disease of aging. Symptoms include bradykinesia and akinesia which are, respectively, slowness of movement and difficulty of movement initiation, increased muscle tone, denoting tight and rigid muscles, and tremors of the hands and jaw. Additionally, as the disease progresses, patients experience deficiencies in cognition \cite{bear_neuroscience_2016}. The prevalence of this neurological condition is rapidly increasing while, currently, around 10 million people worldwide are affected \cite{noauthor_what_nodate}. This highlights the need to establish a widely-accessible and reliable device and method to measure gait.

There are several technologies available to evaluate gait and they can be classified in two distinct groups: non-wearable sensors and wearable sensors   \cite{muro-de-la-herran_gait_2014}. Some example of non-wearable sensors are camera-based motion capturing systems and light detection and ranging sensors (LiDAR). While these techniques are very well adapted for whole body analysis and provide precise information, their implementation is limited to controlled laboratory facilities for a short duration of time and limited by their typical high-cost. On the other hand, wearable sensors have the advantage of continuously tracking a patient during their daily activities and the advantage of their relative affordability \cite{boutaayamou_development_2015}. Some examples of wearable sensors are inertial measurement units (IMUs) such as accelerometers.

As the name suggests, accelerometers measure linear acceleration, the change in linear velocity. There are several different technologies that can be used in an accelerometer, however, they rely on the same principle of measuring the deflections of elements holding a known mass \cite{noauthor_what_2023}.

Given the merit of accelerometers as presented above, the aim of this study is to validate the use of accelerometers as a simple step counter and to calculate temporal gait parameters of healthy participants in order to compare with data from a Parkinson's disease patient.

